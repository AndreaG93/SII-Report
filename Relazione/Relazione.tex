\documentclass[10pt,a4paper, titlepage]{report}

% Pacchetti...
% ----------------------------------------------------------------------------------------- %
\usepackage[T1]{fontenc}
\usepackage[utf8]{inputenc}
\usepackage[italian]{babel}
\usepackage{booktabs}
\usepackage{amsmath}
\usepackage{amsfonts}
\usepackage{amssymb}
\usepackage{listings}
\usepackage{xcolor}
\usepackage{graphicx}
\usepackage{sidecap}
\usepackage{float}
\usepackage{siunitx}
\usepackage{multirow}
\usepackage{hyperref}
\usepackage{bmpsize}
\usepackage{adjustbox}
\usepackage{footnote}
\usepackage{authblk}


% Usato per personalizzare l'ambiente 'listings'...
% ----------------------------------------------------------------------------------------- %
\lstset{
language=C,
basicstyle=\small\ttfamily,			
keywordstyle=\color{blue},
commentstyle=\color{gray},			
stringstyle=\color{black},			
numbers=left,						
numberstyle=\tiny,					
stepnumber=1,						
breaklines=true						
}

% Usato per aggiugnere la numerazione alle sezioni di tipo 'subsubsection' e inserirle nell'indice...%
\setcounter{secnumdepth}{3}
\setcounter{tocdepth}{3}

% Frontespizio...
% ----------------------------------------------------------------------------------------- %
\title{Progetto del corso di Sicurezza informatica e Internet A.A. 2017-2018}

\author[1]{Andrea Graziani (0273395)}
\author[1]{Alessandro Boccini (0277414)}
\author[1]{Ricardo Gamucci (0274716)}
\affil[1]{Università degli Studi di Roma Tor Vergata}


\date{27 marzo 2019}

% Inzio documento...
% ----------------------------------------------------------------------------------------- %
\begin{document}

\maketitle
\tableofcontents
\newpage

\chapter{Analisi tecnica del malware}

\section{Analisi dei file}

\begin{table}[h!]
  \begin{center}
    \caption{Lista dei file facentiff parte del malware FASTCash}
    \centering
    \label{tab:FilesList}
    
    \begin{adjustbox}{width=1.5\textwidth,center=\textwidth}
 
    \begin{tabular}{l|c|r}
      \toprule
      \textbf{Nome} & \textbf{SHA256} \\
      \midrule
      
      Lost\_File.so & \texttt{10ac312c8dd02e417dd24d53c99525c29d74dcbc84730351ad7a4e0a4b1a0eba} \\
      
      Unpacked\_dump\_4a740227eeb82c20... & \texttt{10ac312c8dd02e417dd24d53c99525c29d74dcbc84730351ad7a4e0a4b1a0eba} \\
  
  Lost\_File1\_so\_file & \texttt{3a5ba44f140821849de2d82d5a137c3bb5a736130dddb86b296d94e6b421594c} \\
    
      4f67f3e4a7509af1b2b1c6180a03b3... & \texttt{4a740227eeb82c20286d9c112ef95f0c1380d0e90ffb39fc75c8456db4f60756} \\ 
      
      5cfa1c2cb430bec721063e3e2d144f... & \texttt{820ca1903a30516263d630c7c08f2b95f7b65dffceb21129c51c9e21cf9551c6} \\
      
      Unpacked\_dump\_820ca1903a305162... & \texttt{9ddacbcd0700dc4b9babcd09ac1cebe23a0035099cb612e6c85ff4dffd087a26} \\
      
      8efaabb7b1700686efedadb7949eba... & \texttt{a9bc09a17d55fc790568ac864e3885434a43c33834551e027adb1896a463aafc} \\
      
      d0a8e0b685c2ea775a74389973fc92... & \texttt{ab88f12f0a30b4601dc26dbae57646efb77d5c6382fb25522c529437e5428629} \\
      
      2.so & \texttt{ca9ab48d293cc84092e8db8f0ca99cb155b30c61d32a1da7cd3687de454fe86c} \\
      
      Injection\_API\_executable\_e & \texttt{d465637518024262c063f4a82d799a4e40ff3381014972f24ea18bc23c3b27ee}\\
      
      Injection\_API\_log\_generating\_s & \texttt{e03dc5f1447f243cf1f305c58d95000ef4e7dbcc5c4e91154daa5acd83fea9a8}\\
      
      inject\_api & \texttt{f3e521996c85c0cdb2bfb3a0fd91eb03e25ba6feef2ba3a1da844f1b17278dd2}\\
      
      \bottomrule
    \end{tabular}
    \end{adjustbox}
  \end{center}
\end{table}






















\newpage
\subsection{Analisi del file \texttt{2.so}}

In base all'output ottenuto dal tool unix \texttt{file}, \texttt{2.so} è un file di tipo \textbf{eXtended COFF} (\textbf{XCOFF}) che rappresenta la versione migliorata ed estesa del formato \textbf{Common Object File Format} (\textbf{COFF}), il formato di file standard che ha definito la struttura dei file eseguibili e delle librerie nei sistemi operativi UNIX\footnote{Cfr. \texttt{https://it.wikipedia.org/wiki/COFF}} fino al 1999\footnote{Cfr. \texttt{https://en.wikipedia.org/wiki/Executable\_and\_Linkable\_Format}}, anno della definitiva adozione dello standard \textbf{Executable and Linkable Format} o \textbf{ELF}.
XCOFF rappresenta tuttavia uno standard proprietario sviluppato da IBM\footnote{Cfr. IBM - \textit{XCOFF Object File Format} - \texttt{https://www.ibm.com/support/knowledgecenter/ssw\_aix\_72/com.ibm.aix.files/XCOFF.htm} (data ultima consultazione 27-03-2019)} adottato nei sistemi operativi \textbf{Advanced Interactive eXecutive} o \textbf{AIX}, una famiglia di sistemi operativi proprietari basati su Unix sviluppati dalla stessa IBM.\footnote{Cfr. \texttt{https://www.ibm.com/it-infrastructure/power/os/aix}}

In accordo alle nostre analisi, confermate anche dal report AR18-275A della NCCIC, il file \texttt{file 2.so} rappresenta una \textbf{shared library} che esporta una serie di metodi che consentono l'iterazione con i sistemi finanziari che utilizzano il protocollo \textbf{ISO8583}.\footnote{Cfr. The National Cybersecurity and Communications Integration Center’s (NCCIC), \textit{Malware Analysis Report (AR18-275A)} - 2 Ottobre 2018 - \texttt{https://www.us-cert.gov/ncas/analysis-reports/AR18-275A)}}

\begin{table}[h!]
  \begin{center}
    \caption{Dettagli del file \texttt{2.s0}}
    \centering
    \label{tab:table1}
    
    \begin{adjustbox}{width=1.5\textwidth,center=\textwidth}
 
    \begin{tabular}{l|c|r}
      \toprule
      \textbf{Descrizione} & \textbf{Valore} & \textbf{Comando Unix} \\
      \midrule
      
      \textbf{Nome} & \texttt{2.so} & \texttt{stat -c "\%n" 2.so} \\
      
      \textbf{Dimensione (\textit{byte})} & \texttt{110592} & \texttt{stat -c "\%s" 2.so} \\
   
      \textbf{Data ultima modifca} & \texttt{2018-11-09 11:08:40.000000000 +0100} & \texttt{stat -c "\%y" 2.so} \\
   
      \textbf{Tipo di file} & \texttt{64-bit XCOFF executable or object module} & \texttt{file 2.so} \\
    
      \textbf{MD5} & \texttt{b66be2f7c046205b01453951c161e6cc} & \texttt{md5sum 2.so} \\ 
 
      \textbf{SHA1} & \texttt{ec5784548ffb33055d224c184ab2393f47566c7a} & \texttt{sha1sum 2.so} \\ 
     
      \multirow{2}{*}{\textbf{SHA256}} & \texttt{ca9ab48d293cc84092e8db8f0ca99cb1} & \multirow{2}{*}{\texttt{sha256sum 2.so}} \\ 
      & \texttt{55b30c61d32a1da7cd3687de454fe86c} & \\
      
      \multirow{2} {*}{\textbf{SHA512}} & \texttt{6890dcce36a87b4bb2d71e177f10ba27f517d1a53ab02500296f9b3aac021810} & \multirow{2}{*}{\texttt{sha512sum 2.so}} \\ 
      & \texttt{7ced483d70d757a54a5f7489106efa1c1830ef12c93a7f6f240f112c3e90efb5} & \\
      
      \bottomrule
    \end{tabular}
    \end{adjustbox}
  \end{center}
\end{table}

\subsubsection{Stringhe stampabili rilevanti}\label{strings2so}

Per estrazione di tutte le stringhe stampabili contenute nel file \texttt{2.so} ci siamo serviti del tool \texttt{strings}\footnote{Cfr. \texttt{https://linux.die.net/man/1/strings}} di cui  riportiamo frammenti dell'output ottenuto nei listati \ref{code:strings2so-1} e \ref{code:strings2so-2}.

\begin{lstlisting}[
frame=lines, 
caption={Stringe estratte dal file \texttt{2.so}}, 
label={code:strings2so-1},
firstnumber=465
]
...
_GLOBAL__FI_eg64_so
_GLOBAL__FD_eg64_so
=s4m
/opt/freeware/lib/gcc/powerpc-ibm-aix6.1.0.0/4.2.0/ppc64:/opt/freeware/lib/gcc/powerpc-ibm-aix6.1.0.0/4.2.0:/opt/freeware/lib/gcc/powerpc-ibm-aix6.1.0.0/4.2.0/../../..:/usr/lib:/lib
libc.a
shr_64.o
libpthread.a
shr_xpg5_64.o
...
\end{lstlisting}

Poiché nei sistemi operativi AIX la directory all'interno del quale sono contenute tutte le librerie di GCC assume la forma mostrata nel listato \ref{code:example-1}\footnote{http://www.perzl.org/aix/index.php\%3Fn\%3DMain.GCCBinariesVersionNeutral}, possiamo dedurre dalla riga 496 del listato \ref{code:strings2so-1} che la versione di GCC utilizzata è stata la 4.2.0 (versione rilasciata il 13 Maggio 2007\footnote{http://www.gnu.org/software/gcc/gcc-4.2/}) mentre la versione del sistema operativo bersaglio fosse stata la V6.1, versione ormai obsoleta del sistema operativo AIX il cui supporto è terminato ufficialmente il 30 Aprile del 2017.\footnote{https://www-01.ibm.com/support/docview.wss?uid=swg21634678\#AIX}
Dalla stessa riga osserviamo che l'architettura hardware del sistema bersaglio è equipaggiata con un processore PowerPC

Ovviamente il riferimento alla libreria standard \texttt{libc.c} e di GCC suggeriscono che il malware è stato scritto in C/C++.

\begin{lstlisting}[
frame=lines, 
caption={Formato del percorso di installazione delle librerie GCC nei sistemi operativi AIX}, 
label={code:example-1},
]
/opt/freeware/lib/gcc/<architecture_AIX_level>/<GCC_Level>
\end{lstlisting}

Il listato \ref{code:strings2so-2} mostra ciò che dovrebbero essere i nomi delle procedure esportate dalla libreria il che dimostra in modo inequivocabile il fatto che il malware è in grado di interagire con i sistemi informatici che fanno uso del protocollo ISO8583.

\begin{lstlisting}[
frame=lines, 
caption={Stringhe estratte dal file \texttt{2.so}}, 
label={code:strings2so-2},
firstnumber=545
]
...
DL_ISO8583_MSG_Init
DL_ISO8583_MSG_Free
DL_ISO8583_MSG_SetField_Str
DL_ISO8583_MSG_SetField_Bin
DL_ISO8583_MSG_RemoveField
DL_ISO8583_MSG_HaveField
DL_ISO8583_MSG_GetField_Str
DL_ISO8583_MSG_GetField_Bin
DL_ISO8583_MSG_Pack
DL_ISO8583_MSG_Unpack
DL_ISO8583_MSG_Dump
_DL_ISO8583_MSG_AllocField
DL_ISO8583_COMMON_SetHandler
 DL_ISO8583_DEFS_1987_GetHandler
 DL_ISO8583_DEFS_1993_GetHandler
_DL_ISO8583_FIELD_Pack
_DL_ISO8583_FIELD_Unpack
...
\end{lstlisting}

\subsubsection{Analisi assembler}

Non avendo a disposizione alcuna macchina equipaggiata con un processore 



\newpage

\subsection{Analisi del file \texttt{Injection\_API\_executable\_e}}

In questa sezione dimostreremo come il file di tipo \textbf{eXtended COFF} denominato \texttt{Injection\_API\_executable\_e} sia in grado di eseguire un attacco di \textbf{code injection} a danno di un processo in esecuzione in modo tale da modificarne il comportamento a favore degli attaccanti. 

\begin{table}[h!]
  \begin{center}
    \caption{Dettagli tecnici del file \texttt{2.s0}}
    \centering
    \label{tab:table2}
    
    \begin{adjustbox}{width=1.5\textwidth,center=\textwidth}
 
    \begin{tabular}{l|c|r}
      \toprule
      \textbf{Descrizione} & \textbf{Valore} & \textbf{Comando Unix} \\
      \midrule
      
      \textbf{Nome} & \texttt{2.so} & \texttt{stat -c "\%n" 2.so} \\
      
      \textbf{Dimensione (\textit{byte})} & \texttt{89088} & \texttt{stat -c "\%s" 2.so} \\
   
      \textbf{Data ultima modifca} & \texttt{2018-11-09 11:08:40.000000000 +0100} & \texttt{stat -c "\%y" 2.so} \\
   
      \textbf{Tipo di file} & \texttt{64-bit XCOFF executable or object module} & \texttt{file 2.so} \\
    
      \textbf{MD5} & \texttt{b3efec620885e6cf5b60f72e66d908a9} & \texttt{md5sum 2.so} \\ 
 
      \textbf{SHA1} & \texttt{274b0bccb1bfc2731d86782de7babdeece379cf4} & \texttt{sha1sum 2.so} \\ 
     
      \multirow{2}{*}{\textbf{SHA256}} & \texttt{d465637518024262c063f4a82d799a4e} & \multirow{2}{*}{\texttt{sha256sum 2.so}} \\ 
      & \texttt{40ff3381014972f24ea18bc23c3b27ee} & \\
      
      \multirow{2} {*}{\textbf{SHA512}} & \texttt{a36dab1a1bc194b8acc220b23a6e36438d43fc7ac06840daa3d010fddcd9c316} & \multirow{2}{*}{\texttt{sha512sum 2.so}} \\ 
      & \texttt{8a6bf314ee13b58163967ab97a91224bfc6ba482466a9515de537d5d1fa6c5f9} & \\
      
      \bottomrule
    \end{tabular}
    \end{adjustbox}
  \end{center}
\end{table}

\subsubsection{Stringhe stampabili rilevanti}

Cominciamo lo studio del file \texttt{Injection\_API\_executable\_e} partendo dall'analisi delle stringhe stampabili estratte attraverso il tool \texttt{strings}.
Seguendo lo stesso ragionamento precedentemente descritto nella sezione \ref{strings2so}, possiamo osservare dal listato \ref{code:stringsInjectionAPIexecutablee1} la versione di GCC e del sistema operativo AIX utilizzati per eseguire la \textit{build} del malware, che risultano essere rispettivamente pari a 4.8.5 (la data pubblicazione risale al 23 giugno 2015\footnote{Cfr. https://gcc.gnu.org/gcc-4.8/}), e 7.1 (commercializzata a partire da settembre 2010).\footnote{Cfr: https://www-01.ibm.com/support/docview.wss?uid=isg3T1012517}
Sfortunatamente non è stato possibile risalire alla versione degli aggiornamenti, che IBM identifica con il nome di \textit{Technology Levels} (TLs) \footnote{http://ibmsystemsmag.com/aix/tipstechniques/migration/oslevel\_versions/}, installati sul sistema operativo bersaglio al momento dell'attacco in modo tale da conoscere l'entità del rischio a cui si sottoponeva il sistema bancario. In ogni caso il supporto ufficiale per la versione 7.1, sostituita dalla ben più moderna versione 7.2 rilasciato nel dicembre 2015, è già terminato il 30 Novembre 2013 benché la versione 7.1 TL5 riceverà supporto fino ad aprile 2022.\footnote{Cfr: https://www-01.ibm.com/support/docview.wss?uid=isg3T1012517}

\begin{lstlisting}[
frame=lines, 
caption={Stringhe estratte dal file \texttt{Injection\_API\_executable\_e}}, 
label={code:stringsInjectionAPIexecutablee1},
firstnumber=347]
...
/opt/freeware/lib/gcc/powerpc-ibm-aix7.1.0.0/4.8.5/ppc64:/opt/freeware/lib/gcc/powerpc-ibm-aix7.1.0.0/4.8.5:/opt/freeware/lib/gcc/powerpc-ibm-aix7.1.0.0/4.8.5/../../..:/usr/lib:/lib
...
\end{lstlisting}

Un altro riferimento ai tool utilizzati dagli attaccanti lo possiamo ricavare dalla riga 944 riportata nel listato \ref{code:stringsInjectionAPIexecutablee2} dove apprendiamo l'utilizzo del compilatore lo \textbf{XL C/C++ for AIX} versione 11.1.0.1, quest'ultimo appositamente ottimizzato dalla IBM per i propri sistemi operativi. Ci risulta che tale versione del compilatore non fosse disponibile per AIX 7.1 al momento del rilascio e che sia divenuto disponibile in seguito ad un aggiornamento. \footnote{https://www-01.ibm.com/support/docview.wss?uid=swg21326972}

\begin{lstlisting}[
frame=lines, 
caption={Stringhe estratte dal file \texttt{Injection\_API\_executable\_e}}, 
label={code:stringsInjectionAPIexecutablee2},
firstnumber=944]
...
IBM XL C for AIX, Version 11.1.0.1
...
\end{lstlisting}

Analizziamo ora in dettaglio le varie operazioni compiute dal malware durante la sua esecuzione.
Innanzitutto, osservando la particolare configurazione dei listati successivi come la numero \ref{code:stringsInjectionAPIexecutablee3}, notiamo quello che dovrebbe essere un insieme di stampe nella forma \texttt{[FUNCTION_NAME] [...]} eseguito probabilmente da un meccanismo di log, il che è stato confermato dalla già citata analisi della NCCIC; infatti, notiamo un gran numero di stringhe contenenti i ben noti \textit{conversion specifier} utilizzati nelle stringhe che specificano il formato delle stampe eseguite dalla funzione \texttt{fprintf} di cui molti sono nella forma \%llX, usata per stampare numeri interi senza segno in forma esadecimale\footnote{Cfr. http://man7.org/linux/man-pages/man3/printf.3.html}. 
Come è stato confermato da altre analisi, le stringhe riportate nelle righe 333, 334 e 335 suggeriscono come l'applicazione è stata progettata per essere una command-line utility interattiva e di come il meccanismo di log sia stato utilizzato per ottenere informazioni e consentire agli attaccanti un attacco mirato.

\begin{lstlisting}[
frame=lines, 
caption={Stringhe estratte dal file \texttt{Injection\_API\_executable\_e}}, 
label={code:stringsInjectionAPIexecutablee3},
firstnumber=320]
...
[main] Inject Start
[main] SAVE REGISTRY
[main] proc_readmemory fail
[main] toc=%llX
[main] path::%s
[main] data(%p)::%s
[main] Exec func(%llX) OK
[main] Exec func(%llX) fail ret=%X
[main] Inject OK(%llX)
[main] Inject fail ret=%llX
[main] Eject OK
[main] Eject fail ret=%llX
Usage: injection pid dll_path mode [handle func toc]
       mode = 0 => Injection
       mode = 1 => Ejection
[main] handle=%llX, func=%llX, toc=%llX
[main] ERROR::g_pid(%X) <= 0
[main] ERROR::load_config fail
[main] ERROR::eject & argc != 7
[main] ERROR::g_dl_handle(%llX) <= 0
[main] WARNING::func_addr(%llX), toc_addr(%llX)
...
\end{lstlisting}






Dalle stringhe riportate nel listato \ref{code:stringsInjectionAPIexecutablee4} possiamo intuire la presenza di una ipotetica funzione \texttt{out\_regs} utilizzata dagli attaccanti per stampare forse su un file di log il contenuto dei seguenti registri del processore:
\begin{itemize}
\item GPRs (General Purpose Registers)
\item IAR (Instruction Address Register)
\item MSR (Machine State Register)
\item LR (Link Register)
\item CR (Condition Register)
\item CTR (Control Register)
\item GPRs (General Purpose Registers)
\end{itemize}

\begin{lstlisting}[
frame=lines, 
caption={Stringhe estratte dal file \texttt{Injection\_API\_executable\_e}}, 
label={code:stringsInjectionAPIexecutablee4},
firstnumber=320]
[out_regs] IAR=%llX
[out_regs] MSR=%llX
[out_regs] CR=%llX
[out_regs] LR=%llX
[out_regs] CTR=%llX
[out_regs] GPR%d=%llX
\end{lstlisting}

Come si può apprendere dalla documentazione ufficiale fornita dalla IBM, ogni informazione riguardante un processo con identificatore \texttt{pid} è rappresentata da un grande insieme di file contenuti nella directory \texttt{/proc/pid}\footnote{pag 244} tra cui ricordiamo:

\begin{description}
\item[/proc/pid/status] Questo file riporta lo \textbf{stato} del processo \texttt{pid}.
\item[/proc/pid/ctl] Rappresenta il \textbf{Control File} del processo \texttt{pid} ed è usato per manipolare l'esecuzione del processo \texttt{pid}.
\item[/proc/pid/as] Rappresenta l'Address space del processo \texttt{pid}.
\end{description}

Nel listato ... si apprende come il malware ricostruisce tali percorsi con una chiamata sprintf (lo sappiamo perché è stata trovato un riferimento nel file) e quindi il malware è in grado di conoscere lo stato del processo attaccato, di manipolarne l'esecuzione e modificarne la memoria.

\begin{lstlisting}[
frame=lines, 
caption={Stringhe estratte dal file \texttt{Injection\_API\_executable\_e}},
firstnumber=320]
/proc/%d/ctl
/proc/%d/status
/proc/%d/as
\end{lstlisting}

Il contenuto presente nel listato \ref{code:stringsInjectionAPIexecutablee5} indica come l'attacco preveda la sospensione del processo bersaglio contro il quale eseguire la code injection: dalla documentazione ufficiale AIX si apprende come l'esecuzione dei processi possa essere alterata scrivendo opportuni messaggi
 all'interno di particoali file chiamti \textbf{ctl} (\textit{control}) e \textbf{lwpctl} (\textit{thread control})\footnote{Cfr IBM - \textit{AIX Version 6.1:Reference File} - pp. 230}. Tutti i messaggi di controllo sono descritti da un numero intero che ne identifica l'operazione seguita da altri operandi numerici se previsti 
 \footnote{Cfr IBM - \textit{AIX Version 6.1:Reference File} - pp. 239}. In particolare esiste il comando PCSTOP che permette di arrestare i thread di un particolare processo. Presumibilmente nel listato viene stampato l'identificatore del processo a cui è stato inviato il messaggio PCWSTOP.
 
%ftp://public.dhe.ibm.com/systems/power/docs/aix/61/aixfiles\_pdf.pdf

\begin{lstlisting}[
frame=lines, 
caption={Stringhe estratte dal file \texttt{Injection\_API\_executable\_e}}, 
label={code:stringsInjectionAPIexecutablee5},
firstnumber=319]
...
[proc_wait] PCWSTOP pid=%d, ret=%d, err=%d(%s)
[proc_wait] tid=%d, why=%d, what=%d, flag=%d, sig=%d
...
\end{lstlisting}

Dalla documentazione ufficiale si può facilmente apprendere come l'operatore PCRUN, di cui possiamo notarne il riferimento nel  listato \ref{code:stringsInjectionAPIexecutablee6}, possa essere utilizzato per riavviare l'esecuzione di un processo dopo essere stato arrestato. Ciò indica come l'attacco preveda il riavvio del processo dopo aver eseguito la \textit{code injection}.

\begin{lstlisting}[
frame=lines, 
caption={Stringhe estratte dal file \texttt{Injection\_API\_executable\_e}}, 
label={code:stringsInjectionAPIexecutablee6},
firstnumber=308]
[proc_continue] PCRUN pid=%d, arg=%d, ret=%d, err=%d(%s)
\end{lstlisting}

Infatti come dimostra in modo inequivocabile il listato, il malware è in grado di leggere la memoria allocata dal sistema operativo di un processo e di modificarla per effettuare la code injection vera e propria.

\begin{lstlisting}
[proc_readmemory] ret=%d, err=%d(%s), addr=%p, len=%d, data=%p
[proc_readmemory] (%X~%X) %02X %02X %02X %02X %02X %02X %02X %02X %02X %02X %02X %02X %02X %02X %02X %02X
[proc_writememory] ret=%d, err=%d(%s), addr=%p, len=%d, data=%p
\end{lstlisting}

La manipolazione del processo bersaglio avviene per mezzo di una serie di segnali tra cui:


\begin{description}

\item[PCSET] sets one or more modes of operation for the traced process. 
\item[PCRUN] Riesegue un thread dopo essere stato arrestato; l'operando è un set di flag contenuto in un int.
\item[PCSENTRY] Instructs the process's threads to stop on entry to specified system calls. T
\item[PCSFAULT]Defines a set of hardware faults to be traced in the process. When incurring one of these faults, a thread stops.

\end{description}


\begin{lstlisting}
[proc_attach] PCSET pid=%d, ret=%d, err=%d(%s)
[proc_attach] PCSTOP pid=%d, ret=%d, err=%d(%s)
[proc_attach] PCSTRACE pid=%d, ret=%d, err=%d(%s)
[proc_attach] PCSFAULT pid=%d, ret=%d, err=%d(%s)
[proc_attach] PCSENTRY pid=%d, ret=%d, err=%d(%s)
[proc_detach] PCSTRACE pid=%d, ret=%d, err=%d(%s)
[proc_detach] PCSFAULT pid=%d, ret=%d, err=%d(%s)
[proc_detach] PCSENTRY pid=%d, ret=%d, err=%d(%s)
[proc_detach] PCRUN pid=%d, ret=%d, err=%d(%s)
\end{lstlisting}


\subsubsection{Disassemblaggio}

\begin{lstlisting}
bl      0x10000674 <.out_log>
bl      0x10001220 <.proc_attach>
li      r3,0
bl      0x10001a28 <.proc_continue>
li      r3,0
li      r4,0
bl      0x10001b44 <.proc_wait>
ld      r3,728(r2)
bl      0x10000674 <.out_log>
addi    r9,r31,152
mr      r3,r9
bl      0x10001ee4 <.proc_getregs>
addi    r9,r31,152
mr      r3,r9
bl      0x10000c80 <.out_regs>
addi    r8,r31,536
addi    r10,r31,152
li      r9,384
mr      r3,r8
mr      r4,r10
mr      r5,r9
bl      0x1000324c <.memmove>
nop
ld      r9,536(r31)
addi    r9,r9,-16
mr      r3,r9
li      r4,16384
bl      0x10000b48 <.file_dump>
\end{lstlisting}












La traduzione dal linguaggio macchina all'assembler del file è stato usufrendo del servizio web https://onlinedisassembler.com/ per motivi di semplicità con le seguenti impostazioni

architettura powerpc620
processore POWER 7
64 bit

Queste impostazioni ci hanno permesso di ottenere un output sostanzialmente identico a quello mostrato in vari screenshot dalla CISA




ftp://public.dhe.ibm.com/systems/power/docs/aix/72/idalangref\_pdf.pdf

\begin{table}[h!]
  \begin{center}
    \caption{Dettagli tecnici del file \texttt{2.s0}}
    \centering
    \label{tab:table55}
    
    \begin{tabular}{l|l|l}
      \toprule
        Comando & SIntassi & Descrizione \\
    \midrule
        \texttt{Brach Link (bl)} & bl target\_address & Branches to a specified target address. \\
      
      %\texttt{Condition Register (CR)} &  \\
      
      %\texttt{Link Register (LR)} &  \\
      
      %\texttt{Count Register (CTR)} &  \\
    
 	
      \bottomrule
    \end{tabular}
   
  \end{center}
\end{table}


    LI
    
 	Specifies a 24-bit signed two's-complement integer that is concatenated on the right with 0b00 and sign-extended to 64 bits (PowerPC®) or 32 bits (POWER® family). This is an immediate field.

 	https://www.ibm.com/support/knowledgecenter/ssw\_aix\_72/com.ibm.aix.alangref/idalangref\_inst\_fields.htm
 	
 	mflr r0 \# move LR into GPR0
 	
 	 If a branch instruction has the Link bit set to 1, then the Link Register is altered to store thereturn address for use by an invoked subroutine. The return address is the address of the instructionimmediately following the branch instruction (pag 33)


The following code transfers the execution of the program to here and sets the Link Register: 

https://www.ibm.com/support/knowledgecenter/en/ssw\_aix\_71/com.ibm.aix.alangref/idalangref\_bbranchinst.htm


\listoffigures
\listoftables
\lstlistoflistings

\end{document}













